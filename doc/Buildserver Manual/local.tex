\section{Local Build}
\label{sec:Local}
%\wrappic{0.25}{AntLogo}{Ant Logo}{Ant Logo.}{AntsLogo}{l}
This section explains how to build \EASy\ and all plug-ins on your local hard drive using ANT\footnote{\url{http://ant.apache.org/}}. Section \vref{sec:ANT} roughly explains how to install ANT. Section \vref{sec:Relevant Folders} shows the relevant folders which must be downloaded to run the build scripts. Sections \ref{sec:Preparation of Eclipse} and \ref{sec:global-build.properties} explain local configurations, which must be done before running the build. Finally, Section \vref{sec:Local Build} explains the commands for running the build scripts on a local PC.\\

\subsection{Installation of ANT}
\label{sec:ANT}
Download ANT from \url{http://ant.apache.org/} and extract the archive into the programs folder, e.g. \ttfamily C:\textbackslash Program Files\textbackslash Ant\normalfont. We suggest ANT version 1.8.4, because this version is also used by the build server. Set the \texttt{Path} environment variable to the \texttt{<ANT folder>\textbackslash bin} directory. Figure \vref{fig:antPath} shows an example how to configure the \texttt{Path} variable in Windows.
\pic{0.95}{antPath}{Configuration of the Path variable for ANT}{Configuration of the Path variable for ANT in Windows.}{antPath}

\subsection{Relevant Folders}
\label{sec:Relevant Folders}
The whole \EASy\ tool suite can be downloaded from \EASyURL\ via Subversion. The plug-ins are located in sub folders of \texttt{/trunk}. The documentation build script for the creation of a common JavaDoc is located in \texttt{/doc/javadoc}.

Relevant projects are:
\begin{description}
	\item[/trunk/VarModel] $ $\\
		The variability model and other utility functions which do not depend of other plug-ins.
	\item[/trunk/IVML] $ $\\
		The parser and editor for the variability model.
	\item[/trunk/Instantiation] $ $\\
		Tools and models needed for resolving variability in product line artifacts.
	\item[/trunk/Reasoner/ReasonerCore] $ $\\
		Reasoner core functionality (interfaces and data objects). This package is not able to perform reasonings by its own and needs at least one of the reasoner implementations below:
	\begin{description}
		\item[/trunk/Reasoner/Drools] $ $\\
			A reasoner implementation, using Drools Expert\footnote{\url{http://www.jboss.org/drools/drools-expert.html}}.
		\item[/trunk/Reasoner/Jess] $ $\\
			A reasoner implementation, using Jess\footnote{\url{http://www.jessrules.com/}}. Currently, not maintained.
	\end{description}
	\item[/trunk/EASy-Producer] $ $\\
		\EASy\ core functionality and Eclipse editors.
\end{description}

It is also possible to check out the complete \texttt{/trunk} folder. The result should look like Figure \vref{fig:relevantFolders}.

\label{Copying global-build.properties}
The \texttt{/trunk} folder contains also a \texttt{global-build.properties} file. This file must be copied to the \texttt{HOME} directory, e.g., the \textit{user files} in Windows and edited as described in Section \vref{sec:global-build.properties}.

\pic{.85}{relevantFolders}{Relevant folders for compiling \EASy}{Relevant folders, which must be checked out, for compiling \EASy.}{relevantFolders}

\subsection{Preparation of Eclipse for compilation and testing}
\label{sec:Preparation of Eclipse}
For compilation and testing an Eclipse\footnote{\url{http://www.eclipse.org/}} instance is needed, which provides the plug-ins required by \EASy. The following steps describe the installation process as well as some clean-up to minimize the size of the resulting Eclipse instance:
\begin{description}
	\item[Step 1: Download Eclipse] $ $\\
		Eclipse offers an installer for obtaining a specific Eclipse instance. However, this results in an Eclipse plug-ins directory separated from the main Eclipse directory. Hence, it is required to download a packaged (stand-alone) \textit{Eclipse IDE for Java Developers}. A list of all packages can be found at the Eclipse packages release page\footnote{\url{https://www.eclipse.org/downloads/packages/release}}. Download the desired version as an archive file and extract the included \texttt{eclipse} directory to your preferred location. 
		
		An Eclipse for your operating system is fine although Jenkins runs on Linux. Therefore, Linux specific bundles will have to be added later. For an automatic cross-operating system installation of your Eclipse app, also Mac OS bundles will be required. Thus, it makes sense to download the same Eclipse version for all target operating systems.
	\item[Step 2: Install required plug-ins] $ $\\
		The \textit{Eclipse IDE for Java Developers} already provides basic plug-ins required by \EASy. This basic list of plug-ins is accessible after starting Eclipse and navigating to the installation details via \texttt{Help > About Eclipse IDE > Installation Details}. In this way, one can check, if the additional plug-ins listed below are installed successfully. Further, installing additional plug-ins may require adding new \texttt{Eclipse update sites}. Information about how to add new update sites to an Eclipse instance can be found at the Eclipse help page\footnote{\url{https://help.eclipse.org/2021-03/index.jsp?topic=/org.eclipse.platform.doc.user/tasks/tasks-127.htm}}. The following plug-ins must be installed\footnote{The versions of the added plug-ins are included here for completeness; they may change due to updates in future}:
		\begin{description}
			\item[Xtext] $ $\\
			Update Site: \url{https://download.eclipse.org/modeling/tmf/xtext/updates/composite/releases/} \\
			Selected Feature(s): Xtext > Xtext Complete SDK \\
			Added Plug-in(s): Xtext Complete SDK [2.25.0.v20210301-1429]
			\item[Checkstyle] $ $\\
			Update Site: \url{https://checkstyle.org/eclipse-cs-update-site} \\
			Selected Feature(s): Checkstyle > Eclipse Checkstyle Plug-in \\
			Added Plug-in(s): Eclipse Checkstyle Plug-in [8.40.0.202104250427]
			\item[Eclipse Epsilon] $ $\\
			Update Site: \url{http://download.eclipse.org/epsilon/updates/2.3/} \\
			Selected Feature(s): All, except for category \texttt{Epsilon HTML Integration} \\
			Added Plug-in(s):
			\begin{itemize}
				\item EMC Abstract Spreadsheet Driver [2.3.0.202104221823]
				\item EMC Google Spreadsheet Driver [2.3.0.202104221823]
				\item EMC Google Spreadsheet Driver Development Tools [2.3.0.202104221823]
				\item EMC Microsoft Excel Spreadsheet Driver [2.3.0.202104221823]
				\item EMC Microsoft Excel Spreadsheet Driver Development Tools [2.3.0.202104221823]
				\item EMF Model Comparison for EUnit (EMF Compare 3.x / Kepler and after) [2.3.0.202104221823]
				\item Epsilon Core [2.3.0.202104221823]
				\item Epsilon Core Development Tools [2.3.0.202104221823]
				\item Epsilon CSV Driver [2.3.0.202104221823]
				\item Epsilon CSV Driver Development Tools [2.3.0.202104221823]
				\item Epsilon Development Tools for EMF [2.3.0.202104221823]
				\item Epsilon Development Tools for Simulink [2.3.0.202104221823]
				\item Epsilon Development Tools for UML [2.3.0.202104221823]
				\item Epsilon EMF Integration [2.3.0.202104221823]
				\item Epsilon GraphML Integration [2.3.0.202104221823]
				\item Epsilon Simulink Integration [2.3.0.202104221823]
				\item Epsilon UML Integration [2.3.0.202104221823]
				\item Epsilon Validation Language EMF Integration [2.3.0.202104221823]
				\item Epsilon Wizard Language EMF Integration [2.3.0.202104221823]
				\item Epsilon Wizard Language GMF Integration [2.3.0.202104221823]
				\item Eugenia [2.3.0.202104221823]
				\item Flexmi [2.3.0.202104221823]
				\item Flexmi Development Tools [2.3.0.202104221823]
				\item Human Usable Textual Notation Core [2.3.0.202104221823]
				\item Human Usable Textual Notation Development Tools [2.3.0.202104221823]
				\item Picto [2.3.0.202104221823]
			\end{itemize}
			\item[FindBugs] $ $\\
			Update Site: \url{http://findbugs.cs.umd.edu/eclipse} \\
			Selected Feature(s): FindBugs > FindBugs Feature \\
			Added Plug-in(s): FindBugs Feature [3.0.1.20150306-5afe4d1]
			\item[Graphical Modeling Framework (GMF)] $ $\\
			Update Site: \url{http://download.eclipse.org/modeling/gmp/gmf-tooling/updates/releases} \\
			Selected Feature(s): GMF Tooling > Graphical Modeling Framework (GMF) Tooling SDK \\
			Added Plug-in(s): Graphical Modeling Framework (GMF) Tooling SDK [3.2.1.201409171321]
			\item[Subclipse] $ $\\
			Update Site: \url{https://subclipse.github.io/updates/} \\
			Selected Feature(s): All in category \texttt{Subclipse} and category \texttt{Subclipse - JavaHL for SVN 1.12-1.14+} \\
			Added Plug-in(s):
			\begin{itemize}
				\item Subclipse [4.3.3.202012181204]
				\item Subclipse Integration for Mylyn (Optional) [4.2.0.1]
				\item Subversion 1.14 JavaHL [1.14.0.202005311413]
				\item Subversion Revision Graph [4.2.0.1]
				\item SVNKit Client Adapter [1.8.12.3]
			\end{itemize}
			\item[XML Streaming API] $ $\\
			Update Site: \url{http://download.eclipse.org/tools/orbit/downloads/drops/R20210223232630/repository} \\
			Selected Feature(s): Orbit Bundles By Name javax.* > Java XML Streaming API \\
			Added Plug-in(s): Java XML Streaming API [1.0.1.v201004272200]
			\item[Eclipse Plug-in Development Environment (PDE)] $ $\\
			Update Site: \url{http://download.eclipse.org/eclipse/updates/4.19} \\
			Selected Feature(s): Eclipse Plug-in Development Tools > Eclipse Plug-in Development Environment \\
			Added Plug-in(s): Eclipse Plug-in Development Environment [3.14.700.v20210303-1800]
			\item[Eclipse Test Framework] $ $\\
			Update Site: \url{http://download.eclipse.org/eclipse/updates/4.19} \\
			Selected Feature(s): Eclipse Tests, Tools, Examples, and Extras > Eclipse Test Framework \\
			Added Plug-in(s): Eclipse Test Framework [3.7.1400.v20210222-1948]
		\end{description}
	\item[Step 3: Delete sources] $ $\\
		While this is an optional step, deleting source plug-ins can significantly reduce the size of the Eclipse instance and avoid potential errors when used at the server. For this purpose, all plug-ins (\texttt{*.jar}-files) in the Eclipse plug-in directory (for example, \texttt{eclipse/plugins}) and its sub-directories can be deleted, which have a name that matches the pattern \texttt{".source\_"}. In particular, one must enuure that no \texttt{resource} plug-ins will be deleted as well. Best practice is to cope the plug-ins directory as a backup before starting the deletion and copy that backup into the Eclipse instance again, if after the deletion Eclipse does not work as expected.
\end{description}

The result of the steps above is an Eclipse instance for use at Jenkins, which contains the following installed plug-ins:
\begin{itemize}
	\item Apache Geronimo Activation Source Plug-in
  \item Buildship: Eclipse Plug-ins for Gradle
  \item EclEmma Java Code Coverage
  \item Eclipse Checkstyle Plug-in
  \item Eclipse IDE for Java Developers
  \item Eclipse Java Development Tools
  \item Eclipse Plug-in Development Environment
  \item Eclipse Test Framework
  \item EMC Abstract Spreadsheet Driver
  \item EMC Google Spreadsheet Driver
  \item EMC Google Spreadsheet Driver Development Tools
  \item EMC Microsoft Excel Spreadsheet Driver
  \item EMC Microsoft Excel Spreadsheet Driver Development Tools
  \item EMF Model Comparison for EUnit (EMF Compare 3.x / Kepler and after)
  \item Epsilon Core
  \item Epsilon Core Development Tools
  \item Epsilon CSV Driver
  \item Epsilon CSV Driver Development Tools
  \item Epsilon Development Tools for EMF
  \item Epsilon Development Tools for Simulink
  \item Epsilon Development Tools for UML
  \item Epsilon EMF Integration
  \item Epsilon GraphML Integration
  \item Epsilon Simulink Integration
  \item Epsilon UML Integration
  \item Epsilon Validation Language EMF Integration
  \item Epsilon Wizard Language EMF Integration
  \item Epsilon Wizard Language GMF Integration
  \item Eugenia
  \item FindBugs Feature
  \item Flexmi
  \item Flexmi Development Tools
  \item Git integration for Eclipse
  \item Graphical Modeling Framework (GMF) Tooling SDK
  \item Human Usable Textual Notation Core
  \item Human Usable Textual Notation Development Tools
  \item Java implementation of Git
  \item Java XML Streaming API
  \item JustJ OpenJDK Hotspot JRE Complete
  \item m2e - Maven Integration for Eclipse (includes Incubating components)
  \item m2e - slf4j over logback logging (Optional)
  \item Marketplace Client
  \item Mylyn WikiText Editors
  \item Oomph Setup
  \item Picto
  \item POM Editor using LemMinX language server (includes Incubating components)
  \item Subclipse
  \item Subclipse Integration for Mylyn (Optional)
  \item Subversion 1.14 JavaHL
  \item Subversion Revision Graph
  \item SVNKit Client Adapter
  \item Tip of the Day UI Feature
  \item TM Terminal
  \item Wild Web Developer XML tools
  \item Xtext Complete SDK
\end{itemize}

We suggest to use two instances of such an Eclipse instance at Jenkins:
\begin{itemize}
	\item One unpacked instance for compilation. This Eclipse installation should not contain already compiled \EASy\ plug-ins.
	\item One packed instance for testing. This Eclipse instance will be unpacked, before the test plug-ins will be installed into this instance. This procedure ensures a clean Eclipse installation for testing, which will also contain only needed \EASy plug-ins.
\end{itemize}

\begin{description}
	\item[Step 4: Add the Eclipse test specification files] $ $\\
	
We use \texttt{org.eclipse.test} to run the EASy-Producer unit tests within an Eclipse OSGi environment based on the respective required specified bundles. However, the installation usually misses two files, namely \texttt{library.xml} and \texttt{JUNIT.XLS}. You can obtain both from the actual installation or from\\
\texttt{https://wiki.eclipse.org/Platform-releng/Eclipse\_Test\_Framework}.\\
\texttt{org.eclipse.test} is unpacked during the installation. Copy the two files into the \texttt{org.eclipse.test} folder \texttt{plugins/org.eclipse.test\_...}. 

	\item[Step 5: Archive the test Eclipse] $ $\\

As only Java classes are executed during testing, the installation is now ready for performing the unit tests on Jenkins. The testing Eclipse now contains more bundles than actually required during testing, as currently no UI tests are executed. However, as soon as UI tests are needed, many of the currently unused bundles may be required and it makes sense to have them archived and installed at the point of time when the Eclipse is prepared (as finding the matching bundles later can be a nightmare). Zip the installation so that the archive directly contains the contents of the root folder of Eclipse (not the typical folder \texttt{eclipse}).	

	\item[Step 6: Add cross-operating system bundles] $ $\\
	
Also for compiling EASy-Producer, the test Eclipse from the last step can be used. However, for building installed instances via the Eclipse director application, we use this Eclipse version as repository and then also all operating system specific files required for building the cross-platform bundles are needed. Copy the cross-operating system bundles from the other Eclipse versions downloaded in Step 1 into the plugins folder. You can easily identify them by looking for the substrings ``win'', ``linux'', ``gtj'' or ``cocoa''. A duplication of bundles does not lead to problems as OSGi only selects those relevant for the actual operating system. 

Moreover, for building executable instances, we need the bundle\\
\texttt{org.eclipse.equinox.executable}. If this is not available for your actual Eclipse version, you may try it with the already installed one. Please copy it into the main directory of your Eclipse (not the \texttt{plugins} directory) and also unpack it into a folder named \texttt{binary} in the root folder of your Eclipse so that the starter binaries for the different operating systems are in subfolders of \texttt{binary/bin} (with respective folders for the operating system name, window system etc., e.g., for 64 bit Windows in the subfolder \texttt{win32/win32/x86\_64} (for recent Windows versions still \texttt{win32} is used).

	\item[Step 7: Archive the compile Eclipse] $ $\\

Zip also the installation from step 6.	

	\item[Step 8: Install the Eclipse versions] $ $\\

Copy both archives to Jenkins into the folder \texttt{/var/lib/jenkins/Eclipse}. The test Eclipse from Step 5 remains zipped, the compile Eclipse from Step 7 must be unpacked (so that the unpackt folder directly contains the contents of the root folder of Eclipse, not the typical folder \texttt{eclipse}). 

	\item[Step 9: Test-drive the installation] $ $\\

We recommend to install the compile eclipse first and to review/change the respective settings discussed in Section \vref{sec:global-build.properties}. Potentially it is required to disable the nightly update sites until the compilation and in a second step the tests run stable. As soon as all compile problems (see also Section \vref{sec:FAQ}) are solved, install also the test eclipse, adjust the settings in Section \vref{sec:global-build.properties} and re-run the entire build. If this run is stable, re-activate the nightly update sites (see also Section \vref{sec:FAQ}). 

When the entire build process is running, update the Maven \texttt{EasyDependencies}. All lower-level EASy-Producer components (below the UI) ship with Maven build specifications. However, it is important to understand that typically these build specifications (execpt for some hybrid components) are not used for building rather than the ANT-based approach described here. All those Maven build specifications just declare dependencies, but no build plugins. These Maven specifications are required for the deployment of the lower-level EASy-Producer components and for a flexible component-based integration of EASy-Producer as a library into other projects. As discussed in the developer guidelines, we also offer pre-build EASy-Producer libraries (more or less FAT jars), which may be useful in some settings but not for flexible use of individual EASy-Producer components. 

After an Eclipse update it is important to also update the POM in \texttt{EasyDependencies}, which declares the artifact version of all utilized external artifacts, e.g., those provided by Eclipse (fortunately, this is the case at least since our update to Eclipse 4.19). Typically, it is sufficient to take over the main version numbers from the bundles in the \texttt{plugin} directory. However, if Eclipse bundles are renamed or new artifacts are required, all POM files must be adjusted (not just the one in \texttt{EasyDependencies} that declares the artifact versions).

\end{description}	

\subsection{Editing \texttt{global-build.properties}}
\label{sec:global-build.properties}
The copied \texttt{global-build.properties} (cf. Section \vref{Copying global-build.properties}) must be edited to facilitate local builds. Figure \vref{lst:localSettingOfbuil.properties} shows the relevant entries, which must be configured or at least reviewed/updated when installing a new Eclipse version on Jenkins.

\begin{nofloat}{figure}
	\centering
	\lstinputlisting[language=Properties, firstline=1, lastline=15]{\EASyBuildProperties/global-build.properties}
	\caption[Local settings of the \texttt{global-build.properties}]{Local settings of the \texttt{global-build.properties} (excerpt).}
	\label{lst:localSettingOfbuil.properties}
\end{nofloat}

The entries should be configured as explained below:
\begin{description}
	\item[eclipse.home] $ $\\
		This entry must point to the unpacked Eclipse instance (\textbf{absolute path}), which shall be used for compilation.
	\item[home.base.dir] $ $\\
		This entry must point to the root directory (\textbf{absolute path}) of the downloaded plug-ins, i.e., the downloaded \texttt{/trunk} folder.
	\item[projects.<project>] $ $\\
		These entries must point to the \textbf{relative paths} of the sub folders of the related \texttt{<project>}s inside the \texttt{/trunk} folder.
	\item[emma.path] $ $\\
		This entry must point to the \textbf{absolute path} of the EMMA libraries\footnote{\url{http://emma.sourceforge.net/}}. The development of Emma has been discontinued. As a consequence, we recommend to use the latest release for building \EASy\ plug-ins.
      \item[jacoco.path] $ $\\
             This entry must point to the \textbf{absolute path} of the Jacoco ANT libraries fitting to the JDKs used for testing. There is also a so far undocumented \texttt{jacoco8.path} that pointed for some times to a more recent version - please leave this as it is.
	\item[unzipNewEclipse] $ $\\
		Specification whether a fresh Eclipse instance should be unpacked for each test (\texttt{true}) or not (\texttt{false}).	
	\begin{description}
		\item[alternative.test.eclipse.dir] $ $\\
			\textbf{Absolute path} of an unpacked Eclipse instance, which should be used for testing. Only relevant if \texttt{unzipNewEclipse} was set to \texttt{false}.
	\item[test.eclipse.zip] $ $\\
			\textbf{Absolute path} of an packed Eclipse instance, which should be used for testing. Only relevant if \texttt{unzipNewEclipse} was set to \texttt{true}.
      \item[test.eclipse.testFW.dir] $ $\\
                   \textbf{Relative path} to the unpacked \texttt{org.eclipse.test} bundle installed above in Step 4.
	\end{description}
	\item[eclipse.launcher] $ $\\
	    \textbf{Absolute path} to the Eclipse launcher bundle in the compile Eclipse \texttt{plugins} folder. 
      \item[eclipse.launcher.classpath] $ $\\
           \textbf{Absolute path} to the Eclipse OSGi compatibility state bundle in the compile Eclipse \texttt{plugins} folder. \textit{May be removed in the future}.
      \item[eclipse.executable.jar] $ $\\
           \textbf{Absolute path} to the Eclipse equinox executable installed above in Step 6.
      \item[eclipse.launcher.repo] $ $\\
           \textbf{Absolute path} to the parent folder of the \texttt{native} folder containing unpacked Eclipse executables installed above in Step 6. \textit{May be redirected in the future, e.g., directly to the \texttt{binaries} folder}.
      \item[eclipse.repository] $ $\\
           \textbf{URL} to the Eclipse bundles repository for the headless installation of Eclipse applications using the Eclipse director app. You may add here an official public Eclipse repository, but please be aware that the build process then downloads new bundles upon every build which may lead to unstable and incompatible builds. In contrast, this is set to \textbf{eclipse.home}, which contains all required bundles also for UI-based installations.
\end{description}

\subsection{Building the Projects}
\label{sec:Local Build}
Each of the relevant folders, listed in \vref{sec:Relevant Folders}, contain a \texttt{build.xml} file for building all related plug-ins in one step. All nested plug-ins contain a \texttt{build-jk.xml} for building the single plug-in only.

The build can be started by opening a command shell in the folder an running one of the commands below:
\begin{itemize}
	\item Inside the project dir (e.g. VarModel): \texttt{ant}
	\item Inside the nested plug-in dir (e.g. VarModel/Model): \texttt{ant -f build-jk.xml}
\end{itemize}

Currently, the relationship as shown in Figure \vref{fig:EASy Architecture} exists between the plug-ins (and projects). For this reason, the plug-ins and projects should be build in the opposite order of the \textit{<<use>>} relationships. For instance, the \texttt{Model} must be build before \texttt{IVML} can be build.
\pic{.155}{../../architectureImages/Architecture}{Architecture of \EASy}{The architecture of \EASy.}{EASy Architecture}